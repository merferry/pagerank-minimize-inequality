Algorithmic fairness has attracted significant attention in the past years \cite{tsioutsiouliklis2021fairness}. Given two groups of nodes, we say that a network is fair, if the nodes of the two groups hold equally central positions in the network \cite{pitoura2023pagerank}. Saxena et al. \cite{saxena2022fairsna} note that structural bias of social networks impact the fairness of a number of algorithms used for social-network analysis. Xie et al. \cite{xie2021fairrankvis} present a visual analysis framework for exploring multi-class bias in graph algorithms.

Tsioutsiouliklis et al. \cite{tsioutsiouliklis2021fairness} two classes of fair PageRank algorithms - fairness-sensitive PageRank and locally fair PageRank. They define a stronger fairness requirement, called universal personalized fairness, and show that locally fair algorithms also achieves this requirement. Krasanakis et al. \cite{krasanakis2021applying} present an algorithm for fair ranking with personalization, even when the personalization suffers from extreme bias while maintaining good rank quality. In another paper, Tsioutsiouliklis et al. \cite{tsioutsiouliklis2022link} provide formulae for estimating the role of existing edges in fairness, and for computing the effect of edge additions on fairness. They then propose linear time link recommendation algorithms for maximizing fairness.

% Oostenbach studies fairness in community detection. He proposes three fairness metric for community detection, and evaluates thirty community detection algorithms. He observes that spectral algorithms performs the worst in terms of fairness and accuracy, while other categories have one or more methods that perform well \cite{oostenbachfairness}.

Liang et al. \cite{liang2020equitable} present a connected vehicle based traffic signal control algorithm (at isolated intersections), where they minimize average vehicle delay (a measure of efficiency), while limiting the maximum delay any individual vehicle might have (a measure of equity). Their results show that without the threshold on maximum individual vehicle delay, average delay is often minimized at the expense of very large delays imposed onto some vehicles. By implementing a threshold, both the maximum vehicle delay and the distribution of individual vehicle delays can be improved, often with only negligible impacts to intersection efficiency.

Modeling equity in the allocation of scarce resources is a fast-growing concern in the humanitarian logistics field \cite{alem2022revisiting}.

So far, humanitarian logistics models that have approached equity using the Gini coefficient do not actually optimize its original formulation, but they use alternative definitions that do not necessarily replicate that original Gini measure \cite{alem2022revisiting}. The computational study, based on the floods and landslides in Rio de Janeiro state, Brazil, reveals that while alternative Gini definitions have interesting properties, they can generate vastly different decisions compared to the original Gini coefficient. In addition, viewed from the perspective of the original Gini coefficient, these decisions can be significantly less equitable \cite{alem2022revisiting}.

While the whole society is meant to benefit from sustainable development; environmental and social fairness considerations are often overlooked in the design of supply chain networks\cite{battaia2023environmental}. The literature acknowledges that a reductionist interpretation of sustainability is often found in existing mathematical formulations due to lack of evaluation of the social dimension \cite{battaia2023environmental}.

Battaia et al. \cite{battaia2023environmental} suggest new indicators based on Gini index to measure environmental and social inequalities between regions concerned with installation of new product recovery facilities for the reverse part of a closed-loop supply chain. The initial levels of pollution and unemployment rate in each region are impacted by facility location decisions and corresponding industrial activities. They show through numerical experiments that inequalities between the regions can be diminished as a result of the use of deliberate objectives aiming to reduce environmental and social inequity.

Historically, the main design objective has been economic performance of the supply chain expressed predominantly either as minimization of cost, or as maximization of profit, even if responsiveness and quality were also considered \cite{battaia2023environmental}. With the maturing of the concept of sustainable development, the academic community has undertaken a range of research on sustainable supply chains and their management. Sustainable supply chain research has required the clear definition of performance indicators for environmental and social dimensions of sustainability \cite{battaia2023environmental}.

The Gini index satisfies the principle of Pareto-optimality, which implies that if its value improves, then none of the scores used for its calculation will be worsened \cite{battaia2023environmental}.

Drezner et al. \cite{drezner2009equitable} investigate the location of facilities with equity considerations, namely, minimizing the Gini coefficient of the Lorenz curve based on service distances.

Karsu \cite{karsu2016approaches} consider multi-criteria sorting problems where the decision maker (DM) has equity concerns. In such problems each alternative represents an allocation of an outcome (e.g. income, service level, health outputs) over multiple indistinguishable entities.

Croci et al. \cite{croci2023balanced} consider a bi-objective variant of the $p$-median problem where $p$ facilities must be located to serve a set of $n$ customers with unitary demand. The considered objectives are: minimizing the average traveled distance between customers and facilities, and balancing the number of allocated customers per facility. They denote the latter by customer allocation inequity and measure it as the mean absolute deviation of the number of customers assigned to each median \cite{croci2023balanced}.

Delivering essential pharmaceuticals to consumers in low- and middle-income countries (LMICs) is a complex global challenge that requires equitable solutions in the last mile of pharmaceutical supply chains \cite{bhattacharya2023mathematical}. Bhattacharya et al. \cite{bhattacharya2023mathematical} propose using a mobile pharmacy to alleviate the burden due to pharmaceutical stock-outs among rural communities from an equity lens. They find that optimizing for equity only is associated with high operational costs, and demonstrates approaches that achieve equitable solutions through constrained cost increases.

NOTE: Isn't our process scheduling algorithms following equity. Also consider wait-free algorithms.

There are many applications across a broad range of business problem domains in which equity is a concern and many well-known operational research (OR) problems such as knapsack, scheduling or assignment problems have been considered from an equity perspective \cite{karsu2015inequity}.

Equity is both a technically interesting concept and a substantial practical concern \cite{karsu2015inequity}.

There are various real life applications where equity concerns naturally arise and it is important to address these concerns for the proposed solutions to be applicable and acceptable. As a result, there exist many articles cited in the operational research (OR) literature that consider classical problems, such as location, scheduling or knapsack problems, and extend available models so as to accommodate equity concerns. These models are used across a broad range of applications including but not limited to airflow traffic management, resource allocation, workload allocation, disaster relief, emergency service facility location and public service provision. This broad range of applications indicates that considering these classical models with an emphasis on equity is practically relevant in addition to being technically interesting \cite{karsu2015inequity}.

Equity or fairness, one of the major issues in economics in general, also plays a central role for decision making in humanitarian operations. If, for example, relief commodities are to be supplied to the victims of a natural disaster, aid organizations not only have to take the total degree of demand satisfaction into account, but also the requirement that relief goods should be distributed as equally as possible among the affected population. Ideally, no region or population group should be disadvantaged. The first motivation behind the attempt to organize the distribution in a fair way is of course an ethical one: humans have equal rights and therefore nobody should be discriminated by imbalances in the supply of urgent goods. In humanitarian relief, there is also an additional, more pragmatic motivation for striving for equitable solutions, namely the need to avoid turmoils or assaults by needy people who feel that they are not attended in a fair way by the relief system \cite{gutjahr2018equity}.

Consequently, in the literature on quantitative decision support for humanitarian operations, the equity objective has been included by several authors into their models for optimization or decision making. Two approaches are frequently pursued in these papers (we shall refer to examples in the next section): Either a certain minimal level of equity is ensured by means of an equity constraint, or equity is considered as one of several objectives within a multicriteria optimization model \cite{gutjahr2018equity}.

Gutjahr and Fischer \cite{gutjahr2018equity} observe that practically irrelevant final reductions of average deprivation costs result in substantial increases of inequity, or vice versa, at a low “price of fairness”, dramatic reductions of the Gini inequity index can be achieved.

The build-operate-transfer (BOT) approach is one of the privatization mechanisms for promoting transportation infrastructure developments by using private funds to construct new infrastructure facilities. In a BOT scheme, it often involves three parties: the government, whose objective is to maximize the benefit defined in terms of social welfare added to the society; the private investors, whose objective is to maximize the profit generated from the investment; and the road users, whose objective is to minimize the inequality of benefit distribution among the road users traveling from different origin–destination pairs. Each of these parties has different objectives that often conflict with each other \cite{chen2007analysis}. Chen and Subprasom \cite{chen2007analysis} develop various optimal road pricing models under demand uncertainty for analyzing the tradeoffs among the three objectives.

Equity concerns naturally arise in many real-life applications (e.g., healthcare scheduling, facility location, disaster response operations, air traffic control, etc.), and it is crucial to address these concerns for the proposed solutions to be applicable, equitable, and acceptable. However, accounting for equity is a complex task primarily because there is no unique notion of equity that is generally accepted; the definition of equity often depends on the application \cite{shehadeh2023equity}.

The road network design problems (NDPs) have been addressed in existing literatures from a wide perspective with the emphasis to make an optimal design to mitigate possible externalities, for example, traffic congestion and environmental emissions, in conjunction with road pricing or incentives. However, NDP is inherently multiple objective because of the sensitive characteristics of network performance and the variation of travel behavior. A feasible network design should not only serve the short-term cost–benefit balance but also meet sustainability requirements in which equity issues are regarded as same importance as economic sustainability progress and environmental conservation 1. In urban area, a nearly equal network cost to different destinations for different people indicates an equitable level of spatial distribution. In recent years, both academic research and practical applications on transportation equity have been put forth. Transportation practitioners in the United States, for example, have been advised to avoid disproportionate adverse impacts on minority and low-income groups and to mitigate such impacts when possible \cite{feng2014multicriteria}.

The public service refers to the service provided by federal, state and local government, including health care, education, libraries, parks, fire protection and so on, and the public facilities are established to provide these service over a wide area with spatially distributed demands. On the one hand, the efficiency of the service is important because the resources of the public facilities is limited and their operations are supported (at least partially) by the government tax. On the other hand, different from the commercial environment, the public service should distribute fairly to the designated population to avoid the dissatisfaction of the public and the consequent harm to the development of the society \cite{savas1978equity, zhao2011analyzing}.

The fixed-turn or warabandi system of irrigation management is aimed at providing equitable rationing of Pakistan’s limited water resources. Defining equity as the delivery of an equal depth of water over the irrigated area for a crop season, we find relatively equitable distribution at the distributory level. We identify a need for improved indices that minimize inequity and the difference between canal capacity and operational flows. This is particularly important for canals in the low and lowest priority sub-sets of the warabandi schedule \cite{anwar2013old}.

For many large irrigation systems, distributing water equitably is a stated management objective. Canal operations plans specify which canal to operate at what discharge for each irrigation interval to achieve the stated objective \cite{de2017canal}.

Providing equal access to public service resources is a fundamental goal of democratic societies. Growing research interest in public services (e.g., healthcare, humanitarian relief, elections) has increased the importance of considering objective functions related to equity \cite{yang2013call}.

Resource-allocation problems attempt to distribute limited resources among different subsystems in order to reach the best overall outcome on some defined metric (e.g., profits, costs, throughput). The overall outcome is usually measured by the sum of utility-function values at each subsystem — the efficiency or effectiveness of the entire system. In many service-system settings, equity is also a concern, particularly in public-service environments. For instance, society often desires that resources related to education services, healthcare, disaster relief, and election systems are allocated so that equity among customer groups is maintained. When allocating public-service resources, decision makers care not only about the overall effectiveness in the system, but also, and perhaps more importantly, the equity among the subsystems. Effectiveness and equity can coincide with each other in certain specific situations \cite{butler2002fairness} but more often are in conflict. Historically, equity has received less attention than have effectiveness and efficiency in operations research.

Equity is a classic research topic in a wide range of disciplines from sociology to economics \cite{marsh1994equity}. A rich literature extensively studies the concept and the measures of equity. The oldest research on equity can be traced back to Aristotle, who defined equity as proportional fairness \cite{bertsimas2011price}. In psychology, equity theory was first developed by Adams \cite{adams1965inequity}, who argued that an individual feels it is equitable if the perceived ratio of his/her inputs to outcomes equals that of his/her peers. In political philosophy, Rawls \cite{rawls1971atheory} developed a justice theory — “Justice as Fairness” — and proposed two fundamental justice principles: the liberty principle according to which everyone is equal in front of basic rights and liberties, and the difference principle according to which the wealth of the least-advantaged members should be maximized. In economics, substantial work studies the social welfare distribution with equity issues and investigates numerous metrics related to equity (or equivalently inequity), e.g., see Atkinson \cite{atkinson1970measurement}, Sen and Foster \cite{sen1997economic}, Winfield \cite{winfield2022just}, Silver \cite{silver1989foundations}, Young \cite{young1995equity}. One of the most popular inequity measures in economics is the Gini coefficient \cite{gini1912italian}, defined based on the Lorenz curve and commonly used as an inequity index of income or wealth.

%% The Call for Equity: Simulation-Optimization Models to Minimize the Range of Waiting Times
In the realm of operations research, equitable resource allocation has a wide range of application areas \cite{luss1999equitable} such as facility location, telecommunication networks, air-traffic flow management \cite{bertsimas2011proposal, rios2007delay}, water-resource management \cite{brill1976equity}, and humanitarian relief \cite{beamon2008performance, campbell2008routing, huang2012models}. Facility location is one of the first domains in operations research to consider equity issues \cite{o1969model, mumphrey1971decision, mcallister1976equity, savas1978equity, marsh1994equity}. In these applications, equity was used to determine locations of public-service facilities (e.g., schools, hospitals, libraries, etc.) in unevenly populated areas, because the coverage of these services is influenced by facility locations \cite{coulter1980measuring, erkut1992multiobjective, ogryczak2009inequality}. Keeney \cite{keeney1980equity, keeney1980utility} proposed a risk equity metric to evaluate the risks to individuals due to hazards of normal operations (e.g., a nuclear power plant) or accidents (e.g., a tank car explosion). Telecommunications networks must equitably allocate network resources (e.g, bandwidth) to various types of services at various destination nodes \cite{luo2004packet, ogryczak2002equitable, ogryczak2005telecommunications, radunovic2004rate}. Politics is another area of application for allocating resources equitably. The seats of the United States House of Representatives are allocated to preserve equity — ideally, proportional to the population of each state. The well-known “Alabama paradox” \cite{balinski1983apportioning} demonstrated the counter-intuitive result that an increase in population for a state could actually lead to fewer Congressional seats for that state under commonly used apportionment methods. The introduction of electronic voting technology in 2002 in the United States led to concerns regarding inequitable voter waiting times among precincts. Reports emerged of some voter demographics being treated unfairly due to inequitable allocation methods of voting machines to precincts \cite{mcphee2006ritter, flaherty2008ohio, levine2008excitement}. Models such as those proposed here to improve equity can be applied in many of these scenarios.

%% The Call for Equity: Simulation-Optimization Models to Minimize the Range of Waiting Times
In the general problem of allocating resources equitably, a limited number of resources are to be allocated among different groups of “customers.” We refer to each of these groups of customers as a subsystem. Each subsystem has its own preference measured by a utility function (e.g., waiting time, cost) that depends on the level of resources assigned to it. The decision maker (DM) wishes to allocate resources equitably among the subsystems according to a defined inequity metric over the utility functions of all groups. Similar problems have been formulated as “workload-allocation problems” in traditional manufacturing settings. Here, a DM attempts to allocate resources among subsystems to minimize total holding or waiting costs \cite{fox1966discrete, rolfe1971note, dyer1977note}. Hung \cite{hung2006allocation} and Hung and Posner \cite{hung2007allocation} assign multi-class jobs to multiple servers in M/M/s queueing systems with the objective of minimizing holding costs, but they assume equal service times and they do not capture equity-related issues as we do here. Yang et al. \cite{yang2014improving} explicitly consider equity for a resource-allocation problem related to election systems. There, the focus is on deterministic optimization methods and comparing alternative formulations. Here, we expand the discussion of simulation-optimization methods and include rigorous results for a more robust solution method.
